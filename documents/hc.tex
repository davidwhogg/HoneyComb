\documentclass[12pt, letterpaper, preprint]{aastex}

\newcommand{\project}[1]{\textsl{#1}}
\newcommand{\kepler}{\project{Kepler}}

\newcommand{\frequency}{\nu}
\newcommand{\largedelta}{\Delta\frequency}
\newcommand{\smalldelta}{\delta\frequency}
\newcommand{\windowmean}{\bar{\frequency}}
\newcommand{\windowwidth}{W}

\begin{document}

At the precision of projects like the NASA \kepler\ mission,
the ``observables'' of asteroseismology are limited.
In one sense, they consist of the frequencies, amplitudes, and widths (quality factors)
of a small set of modes found by taking a periodogram of the stellar lightcurve.
These mode properties are statistics of the data---found by taking a periodogram
and either inspecting it or measuring it---but even then, for most applications of asteroseismology,
it is just a few statistics \emph{of} those statistics that are used:
The modes appear like a comb, with a dominant ``large'' frequency spacing $\largedelta$;
within this there are also some shifted modes, shifted by a ``small'' frequency spacing $\smalldelta$;
and the mode amplitudes are modulated by a window function with some central frequency $\windowmean$
and some width $\windowwidth$.

Over-simplifying even further, we care most about the large frequency spacing $\largedelta$.
That is, in a typical project, we take hundreds of thousands of photometric points at huge signal-to-noise,
take a periodogram on a grid of hundreds of thousands of frequencies,
measure the positions and heights of modes,
and from these mode positions measure \emph{one single real-valued quantity}, the large spacing $\largedelta$.
That's horrifying!
Incredibly effective, valuable, and high impact, but also horrifying.

Here we ask the following question:
If \emph{all} we care about is the large spacing $\largedelta$,
can we obtain a measurement of this quantity \emph{without} also measuring the entire periodogram and all modes?
That is, in general you don't want to waste information in the data measuring things you don't care about.
Can we do the most information-preserving possible measurement of the spacing $\largedelta$
directly on the photometric data without going through intermediate steps?
And if we succeed, will this give us the ability to measure the spacing $\largedelta$
with data that are noisier, worse sampled, and cheaper to obtain?
That is, could we greatly reduce the cost and greatly increase the effectiveness of
first-order asteroseismology?

If we find that we \emph{can} reduce the costs of asteroseismology,
it will embolden us to ask more and more crazy questions.
For example, can we imagine a practical project that could measure the large spacing $\largedelta$
for a billion stars?
For another, can we imagine measuring the spacing $\largedelta$ using data where the
integration time is longer than the typical periods corresponding to the typical frequencies or $\windowmean$?
That is, when we are only trying to measure one single number per star,
and nothing \emph{other} than that one number,
there are a lot of things that might become possible that otherwise seem impossible.

\end{document}
