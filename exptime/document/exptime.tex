% This file is part of the HoneyComb project.
% Copyright 2015 the authors.

\documentclass[12pt, preprint]{aastex}

\input{vc}
\newcommand{\giturl}{\url{http://github.com/davidwhogg/HoneyComb}}

\newcommand{\project}[1]{\textsl{#1}}
\newcommand{\TESS}{\project{TESS}}
\newcommand{\LSST}{\project{LSST}}

\newcommand{\dd}{\mathrm{d}}
\newcommand{\transpose}[1]{{#1}^{\mathsf{T}}}
\newcommand{\inverse}[1]{{#1}^{-1}}

\begin{document}
\title{
  Vary all the exposure times!
}
\author{
  David~W.~Hogg\altaffilmark{\CCPP, \CDS, \MPIA, \correspondence},
  Ruth~Angus\altaffilmark{\Oxford, \CfA},
  \&
  Daniel~Foreman-Mackey\altaffilmark{\CCPP, \CDS}
}
\newcommand{\CCPP}{1}
\altaffiltext{\CCPP}{
  Center for Cosmology and Particle Physics,
  Department of Physics,
  New York University
}
\newcommand{\CDS}{2}
\altaffiltext{\CDS}{
  Center for Data Analysis,
  New York University
}
\newcommand{\MPIA}{3}
\altaffiltext{\MPIA}{
  Max-Planck-Institut f\"ur Astronomie,
  Heidelberg
}
\newcommand{\correspondence}{4}
\altaffiltext{\correspondence}{
  \texttt{david.hogg@nyu.edu}
}
\newcommand{\Oxford}{5}
\altaffiltext{\Oxford}{
  Oxford University
}
\newcommand{\CfA}{6}
\altaffiltext{\CfA}{
  Harvard--Smithsonian Center for Astrophysics,
  Harvard University
}
\begin{abstract}
Most astronomical surveys with time-domain astrophysics goals are
aiming to take data as uniformly as possible.
Here we show that as a survey varies its exposure times (from exposure
to exposure), it becomes much more sensitive to stellar variability at
periods shorter than the mean exposure time and shorter than the mean
time between exposures.
When thinking about such short periods, it is improper to think of the
data as ``sampling'' the stellar light curve; the data must be modeled
accurately as being generated by averages or integrations of the light
curve over the individual-exposure exposure times.
We show that---unlike in a survey with uniform exposure times---the
sensitivity to short-period variability of the data in a
variable-exposure-time survey is a very smooth function of period,
degrading monotonically towards the very short-period limit.
Sensitivity to long-period variability is unharmed by randomizing the
exposure times.
Near-future imaging surveys (such as \TESS\ and \LSST) can capitalize
on these results at no cost in total survey duration, data volume, or
bandwidth simply by randomizing exposure times (within a well-defined
interval).
Costs and benefits are discussed critically.
\end{abstract}
\keywords{
  asteroseismology
  ---
  old-school statistics
  ---
  signal processing
  ---
  surveys
  ---
  time-domain sickness
}

\section{Introduction}

Motivated by the \LSST\ straw plan.

No ground-based survey can be homoskedastic, so why go with constant exposure time?

Salutory effects on point-spread function, calibration, dynamic range, non-linearities.

Costs of tracking (in the system and in data analysis) the exposure times used.

Here we consider the asteroseismological angle.

On the costs side: When we vary the exposure time, we have to think of
the telescope or observing program not as sampling the light curve of
the star but rather averaging it, over finite exposure durations.

\section{Generative model}

\begin{eqnarray}
\transpose{y} &\equiv& [y_{1}, y_{2}, \cdots , y_{N}]
\\
y_{n} &=& \int_{t_{1n}}^{t_{2n}} S(t)\,\dd t + \mbox{noise}
\quad.
\end{eqnarray}

\section{Information theory}

\begin{eqnarray}
I_{k} &=& \transpose{m}_{k}\cdot\inverse{C}\cdot m_{k}
\\
\transpose{m}_{k} &\equiv& [m_{k1}, m_{k2}, \cdots , m_{kN}]
\\
m_{kn} &\equiv& \int_{t_{1n}}^{t_{2n}} g_{k}(t)\,\dd t
\\
g_{k}(t) &\equiv& \cos(\omega_{k}\,t + \phi_{k})
\quad .
\end{eqnarray}

\begin{eqnarray}
\cos\alpha_{kk'} &=& \transpose{\hat{m}}_{k}\cdot\inverse{C}\cdot\hat{m}_{k'}
\\
\hat{m}_{k} &\equiv& \frac{m_{k}}{\sqrt{\transpose{m}_{k}\cdot\inverse{C}\cdot m_{k}}}
\quad .
\end{eqnarray}

\section{Data analysis}

\section{Discussion}

\acknowledgements
It is a pleasure to thank Zach Berta-Thompson for helpful information
and discussions.
This manuscript was generated within the \project{HoneyComb}
open-source software project, at git hash
\texttt{\githash~(\gitdate)}.
All code and documentation for this project is available
online\footnote{\texttt{\giturl}} under an open-source license.

\end{document}
